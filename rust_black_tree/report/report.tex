\documentclass[letterpaper]{article}
\synctex=1
\usepackage{graphicx}
\graphicspath{ {images/} }

\usepackage{lipsum}
\usepackage{float}

% \usepackage[
%     style=ieee,
%     backend=biber
%     ]{biblatex}
% \addbibresource{references.bib}

\usepackage{hyperref}

\usepackage{amssymb}

\usepackage{siunitx}

\usepackage{multirow}
% for merging table cells I think

\usepackage{tabularx}
\renewcommand\tabularxcolumn[1]{m{#1}}% for vertical centering text in X column
% allows for linewrap within cells
\newcolumntype{Y}{>{\centering\arraybackslash}X}

\usepackage{todonotes}
\usepackage{pdfpages}

\usepackage{fancyhdr} %header
\fancyhf{}
\renewcommand\headrulewidth{0pt}
\fancyfoot[C]{\thepage}
\renewcommand\footrulewidth{0pt}
\pagestyle{fancy}

\usepackage[pdf]{graphviz}
\usepackage{adjustbox}

\usepackage{amsmath}

% make subsection use letters
%\renewcommand{\thesubsection}{\alph{subsection}}

%\usepackage{minted}

% \usepackage{amsthm}
\title{ECE 421 Project 2\\
Trees, Trees, and More Trees}
\author{Arun Woosaree\\
Alexander Rostron\\
Jacob Reckhard
}

%actual document
\begin{document}

\maketitle %insert titlepage here

\section{Rationale}
From the beginning, we knew there would be a fair amount of shared functionality
for the Red Black tree and the AVL tree. Therefore, special considerations were
taken to ensure a low level of code duplication, in order to adhere to the DRY
(Don't Repeat Yourself) principle.

\subsection{The Node Trait}\label{dry}
Both the Red Black tree and the AVL tree have nodes, but the Red Black Tree's
nodes need to store colour information, while the AVL tree's nodes need to store
depth information. Thus, a Node trait was created, with default implementations
for getting a string representation of the Node, getting the Node's height, its
size, min, side, sibling, uncle, and determining if the node is a child.

To use this trait, a ColorNode struct, and a DepthNode struct were created,
which have implementations for methods defined by the Node trait but do not have
default implementations. Then, the functionality which is specific to either
ColorNode, or the DepthNode is found in the structs' respective implementation
blocks. This approach allowed us to re-use common functionality between the
different types of nodes, while still allowing for new, separate functionality
defined by the different nodes. 

\subsection{The Tree Trait}
For the Tree structs for the Red Black tree and AVL trees, we decided to not go
through the trouble of sharing code. There is a small amount of code
duplication, but because the tree deals with different types of nodes, we deemed
it reasonable to have a little bit of code duplication, like the \texttt{find}
method, for example.
\todo{are we sharing the trait?}

Both the Red Black tree and the AVL tree share a Tree trait, with default
implementations for contains, \dots \todo{}. Then, the functionality which is
specific to either the Red Black tree or the AVL tree is implemented in the Red
Black tree struct, and the AVL tree struct, respectively.This approach allowed
us to re-use common functionality between the
different types of trees, while still allowing for new, separate functionality
defined by the different trees.

\subsection{Arena based allocation}
The way the memory is allocated for the nodes is a little unusual, in that the
nodes are actually stored in a vector. This allows us to deal with indices,
which reduces the amount of pointers we have to deal with. Essentially, there is
a vector in the tree which contains all the nodes, and each node stores its
parent child relationships as an \texttt{Option<usize>}. This also allowed us to
deal with less headaches dealing with lifetimes and ownership.

\subsection{Unsafe}
In order to return a reference to a value of a vector contained within a
refcell, a raw pointer is used. The unsafe code could be avoided by replacing
each call to self.get(n) with \texttt{\&self.data.borrow()[n]} and each call to
\texttt{self.get\_mut(n)} with \texttt{\&mut self.data.borrow()[n]}. This allows
us to do the same thing with less keystrokes. It does make the program not
thread-safe, but self-balancing trees are a pretty questionable choice for a
multi-threaded data structure anyways, since re-balancing can require that most
of the tree be locked to one thread during an insertion or deletion

\section{Testing}
Several unit tests were created to verify the functionality of the Red Black
tree and the AVL trees. 

For both the Red Black tree, and the AVL tree, there are tests to verify:
\begin{enumerate}
  \item the \texttt{contains} function
  \item insertion
  \item printing a string representation of the tree
  \item verifying the tree's height is correct
  \item deletion
\end{enumerate}

Furthermore, tests were created for the \texttt{Node} class. These tests verify
the following functionality:
\begin{enumerate}
  \item getting a children nodes
  \item getting the sibling node
  \item getting uncle nodes
  \item getting the size of a node
  \item getting the height of a node
  \item finding the min value
  \item printing a string representation of the node
\end{enumerate}

\section{Defects}

There are no known faults in the program.
(Or we could mention that the CLI will panic if we haven't handled erroneous
input yet)

\section{Benchmarking}
Both the Red Black tree and the AVL tree were benchmarked against each other.
Overall, the \todo{} tree seemed to be faster.

The results are illustrated below.
\todo{create charts}

\section{Additional features}
We decided to create a macro for instantiating the trees.  This will make it
easier for programmers who decide to use our library for some reason to fill
trees with initial values conveniently.  Similar to how vectors in Rust have a
macro \texttt{vec!}, a Red Black tree, or AVL tree can be created using the
following macros:

\begin{texttt}
  redblack!{1, 2, 3, 4, 5, 6, 7, 8, 9}
\end{texttt}

\begin{texttt}
  avl!{1, 2, 3, 4, 5, 6, 7, 8, 9}
\end{texttt}

\section{Answers to questions}

\textit{What does a red-black tree provide that cannot be accomplished with
ordinary binary search trees?}

Red Black trees, unlike ordinary Binary Search trees, allow for efficient
in-order traversal (the order Left–Root–Right) of their elements.
Because  height of a Red Black tree remains \(O(\log n)\) after every
insertion and deletion, we can then guarantee that the
upper bound for searching, inserting, deleting, etc is also \(O(\log n)\)

In comparison, an unbalanced Binary Search tree might have a worst-case scenario
of a traversal of size \(O(n)\), for example if every single element was
inserted in ascending or descending order such that each child in the tree is
either all on the left side or all on the right side.

\textit{Please add a command-line interface (function main) to your crate to
allow users to test it.}
\todo{print help output of the CLI}

\textit{Do you need to apply any kind of error handling in your system (e.g., panic macro,
Option<T>, Result<T, E>, etc..)}
\todo{i dunno\dots}

\textit{What components do the Red-black tree and AVL tree have in common? Don’t Repeat
Yourself! Never, ever repeat yourself – a fundamental idea in programming.}
The Red Black tree and the AVL tree have a ColorNode, and a DepthNode,
respectively. These nodes share the Node trait, which we have defined.
Furthermore, the Red Black tree and the AVL tree share a Tree trait, which
handles common functionality such as determining if the tree contains a certain
value.
Please see the design rationale above for more details. \ref{dry}
\end{document}
