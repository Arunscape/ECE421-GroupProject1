\documentclass[letterpaper]{article}
\synctex=1
\usepackage{graphicx}
\graphicspath{ {images/} }

\usepackage{lipsum}
\usepackage{float}

% \usepackage[
%     style=ieee,
%     backend=biber
%     ]{biblatex}
% \addbibresource{references.bib}

\usepackage{hyperref}

\usepackage{amssymb}

\usepackage{siunitx}

\usepackage{multirow}
% for merging table cells I think

\usepackage{tabularx}
\renewcommand\tabularxcolumn[1]{m{#1}}% for vertical centering text in X column
% allows for linewrap within cells
\newcolumntype{Y}{>{\centering\arraybackslash}X}

\usepackage{todonotes}
\usepackage{pdfpages}

\usepackage{fancyhdr} %header
\fancyhf{}
\renewcommand\headrulewidth{0pt}
\fancyfoot[C]{\thepage}
\renewcommand\footrulewidth{0pt}
\pagestyle{fancy}

\usepackage[pdf]{graphviz}
\usepackage{adjustbox}

\usepackage{amsmath}

% make subsection use letters
\renewcommand{\thesubsection}{\alph{subsection}}

%\usepackage{minted}

% \usepackage{amsthm}
\title{ECE 421 Project 1\\
Stock Project Monitor}
\author{Arun Woosaree\\
Alexander Rostron\\
Jacob Reckhard
}

%actual document
\begin{document}

\maketitle %insert titlepage here

\section{Rationale}
Since this software interacts with an external entity with API
limitations, we had to make some adjustments to the
specifications. Since the API limits the number of calls per minute,
thus, the software is unable to make enough API calls in under a
minute to gather the required information on 10 stocks. So, we thought
it more correct to stall the program until it received information on
all 10 stocks instead of proceeding with information on a limited
number of stocks. The runtime of this software is primarily bounded by
the API limitations, but does produce the correct result (assuming
stock prices haven't changed during the runtime of the program).

The functional features of Java we used were effectively, maps,
reduces, filters, and lambdas. Thus, the \texttt{pickShareFunctional} method
ends up effectively being a one-liner with no local variables.

In the event that there are two stocks with the highest price under \$500 we
arbitrarily pick one of the stocks and return that.

In the event that the API does not give a response with the price, it is assumed
that the price of the stock is 0. This can happen when an invalid stock is
passed to the function, and can also happen in the event that the API is down or
returns an error response.

\section{Testing}
Unit tests were created to verify two simple scenarios. The first test was for
the normal case, where all of the stock prices returned are different, and we
expect the function to return the stock with the highest price under \$500. For
the second test, we assert that in the case where there is more than one
stock under \$500 with the maximum value, any one of those stocks is returned. 

\section{Defects}

The biggest fault of the program is that there is an API key
limitation in that it can only get 5 stock prices every minute. This
is problematic, because there is 10 stocks the program is to
check. That means it will take at least a minute to run regardless of
how the code is designed. This delay greatly overshadows any other
aspect of the code.

\end{document}
