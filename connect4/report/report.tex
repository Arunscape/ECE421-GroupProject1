\documentclass[letterpaper]{article}
\synctex=1
\usepackage{graphicx}
\graphicspath{ {images/} }

\usepackage{lipsum}
\usepackage{float}

% \usepackage[
%     style=ieee,
%     backend=biber
%     ]{biblatex}
% \addbibresource{references.bib}

\usepackage{hyperref}

\usepackage{amssymb}

\usepackage{siunitx}

\usepackage{multirow}
% for merging table cells I think

\usepackage{tabularx}
\renewcommand\tabularxcolumn[1]{m{#1}}% for vertical centering text in X column
% allows for linewrap within cells
\newcolumntype{Y}{>{\centering\arraybackslash}X}

\usepackage{todonotes}
\usepackage{pdfpages}

\usepackage{fancyhdr} %header
\fancyhf{}
\renewcommand\headrulewidth{0pt}
\fancyfoot[C]{\thepage}
\renewcommand\footrulewidth{0pt}
\pagestyle{fancy}

\usepackage[pdf]{graphviz}
\usepackage{adjustbox}

\usepackage{amsmath}
\usepackage[smartEllipses]{markdown}

% make subsection use letters
%\renewcommand{\thesubsection}{\alph{subsection}}

%\usepackage{minted}

% \usepackage{amsthm}
\title{ECE 421 Project 3\\
Rusty Connect 4}
\author{Arun Woosaree\\
Alexander Rostron\\
Jacob Reckhard
}

%actual document
\begin{document}

\maketitle %insert titlepage here


\section{Design}
\subsection{What can we do on a computer than we can’t do on a printed board?}

\subsection{What is a computerized opponent? What are its objectives?}
- What characteristics should it possess? Do we need an opponent or opponents?


\subsection{What design choices exist for the Interface components?}
- Color? Font? Dimensions of Windows? Rescale-ability? Scroll Bars? \dots

\subsection{What does exception handling mean in a GUI system?}

\subsection{Do we require a command-line interface for debugging purposes?}
Yes, because\dots

\subsection{What are Model-View-Controller and Model-View-Viewmodel? Are either applicable to your
design?}

\section{Innovations (Additional to the specification)}
\begin{enumerate}
    \item Mobile friendliness
    \item Online multiplayer.
        \begin{enumerate}
            \item Users can be on different computers
            \item Users can even leave mid-game and continue later.
            \item Users can leave and continue playing on a different computer
        \end{enumerate}
    \item Secure sign in. Passwords are hashed using argon2. JSON Web tokens are used to ensure a user
        is authorized 
    \item Online deployment. The games are available to play at \url{http://connect4.woosaree.xyz}
    \item Continuous Integration and Deployment. Builds are automatically done when code is pushed,
        which will publish updated images to DockerHub on success \url{https://hub.docker.com/repository/docker/arunscape/connect4-server}
        the updated image is then pulled, and deployed automatically to \url{http://connect4.woosaree.xyz}
    \item Published on crates.io
        
\end{enumerate}

\section{Rationale}

\section{Defects}

\section{Remaining MEAN stack code}
The only thing left over from the MEAN stack is mongodb. Everything else has been written
in Rust and WebAssembly. In a project with over 4800 lines of code,
less than 40 lines of code are HTML + JavaScript. 


\section{User Manual}
\markdownInput{../README.md}
\end{document}