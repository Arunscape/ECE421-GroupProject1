\documentclass[letterpaper]{article}
\synctex=1
\usepackage{graphicx}
\graphicspath{ {images/} }

\usepackage{lipsum}
\usepackage{float}

% \usepackage[
%     style=ieee,
%     backend=biber
%     ]{biblatex}
% \addbibresource{references.bib}

\usepackage{hyperref}

\usepackage{amssymb}

\usepackage{siunitx}

\usepackage{multirow}
% for merging table cells I think

\usepackage{tabularx}
\renewcommand\tabularxcolumn[1]{m{#1}}% for vertical centering text in X column
% allows for linewrap within cells
\newcolumntype{Y}{>{\centering\arraybackslash}X}

\usepackage{todonotes}
\usepackage{pdfpages}

\usepackage{fancyhdr} %header
\fancyhf{}
\renewcommand\headrulewidth{0pt}
\fancyfoot[C]{\thepage}
\renewcommand\footrulewidth{0pt}
\pagestyle{fancy}

\usepackage[pdf]{graphviz}
\usepackage{adjustbox}

\usepackage{amsmath}
\usepackage[smartEllipses]{markdown}

% make subsection use letters
%\renewcommand{\thesubsection}{\alph{subsection}}

%\usepackage{minted}

% \usepackage{amsthm}
\title{ECE 421 Project 3\\
Rusty Connect 4}
\author{Arun Woosaree\\
Alexander Rostron\\
Jacob Reckhard
}

%actual document
\begin{document}

\maketitle %insert titlepage here


\section{Design}
\subsection{What can we do on a computer than we can’t do on a printed board?}
Play against a computer, connect our games to a database for statistics, save a board state for later use, play connect4 and other games with our friends while practicing social distancing, and other things like play validation and easier win condition checking.

\subsection{What is a computerized opponent? What are its objectives?}
- What characteristics should it possess? Do we need an opponent or opponents?


\subsection{What design choices exist for the Interface components?}
- Color? Font? Dimensions of Windows? Rescale-ability? Scroll Bars? \dots

\subsection{What does exception handling mean in a GUI system?}

\subsection{Do we require a command-line interface for debugging purposes?}
Yes, because\dots

\subsection{What are Model-View-Controller and Model-View-Viewmodel? Are either applicable to your
design?}

Model-View-Controller is a software design pattern, commonly used for user interfaces.
The model contains the data structure, the view takes care of how to display the model,
and the controller handles updates to the model and tells the view when to update.

Model-View-Viewmodel is another software design pattern. The model and view is similar to model-view-controller.
However, the viewmodel meaning the view model is responsible for converting the data objects from the model in such
a way that the objects can be easily presented.

Our application uses the Yew framework, a rust web framework based on Elm, which uses the model
view controller architecture. Our web components have a Model struct, a Message enum, and the
struct implementation has a view() function which returns html elements, while the update()
method determines whether or not the component needs to re-render and how the model should be updated
based on the message passed to it.



\section{Innovations (Additional to the specification)}
\begin{enumerate}
    \item Mobile friendliness
    \item Online multiplayer.
        \begin{enumerate}
            \item Users can be on different computers
            \item Users can even leave mid-game and continue later.
            \item Users can leave and continue playing on a different computer
        \end{enumerate}
    \item Secure sign in. Passwords are hashed using argon2. JSON Web tokens are used to ensure a user
        is authorized
    \item Online deployment. The games are available to play at \url{http://connect4.woosaree.xyz}
    \item Continuous Integration and Deployment. Builds are automatically done when code is pushed,
        which will publish updated images to DockerHub on success \url{https://hub.docker.com/repository/docker/arunscape/connect4-server}
        the updated image is then pulled, and deployed automatically to \url{http://connect4.woosaree.xyz}
    \item Published on crates.io

\end{enumerate}

\section{Rationale}

We initially wrote connect4 as a command line app to ensure that games and AI's worked. With important game playing functionality written, we could proceed with the rest of the project. We ensured that the basic types in this connect4 library were serializable through the Serde and bson crates, that way we could easily write those basic connect4 objects to the mongoDB database easily. For example, BSON does not handle usize data types, so many of the connect4 library usizes needed to be re-written as isize instead.

We also strived to make this project as decoupled from the original specifications as we could. We didnt want to spend time reverse-engineering the app, instead we created a list of functionality that we needed to have, and made the app. We also concluded it would not be that much more difficult to strive for a multiplayer application as it was already a web app. We opted for a restful API built with rocket, and other powerful front end modules like yew.

With a strong preference for Rust over Javascript, we tried to write this project with as much Rust code as possible. Looking back, it may have been easier to have simple button presses implemented with basic javascript. There was a major front end refactor due to click decisions among other things.

We had the time, so we decided to implement security features like argon hased passwords and Json Web Tokens to enable a user to remain logged in for a day.


Our dev team likes using apps like Vim, so a lot of this app was developed in the command line. We also used our existing knowledge of Docker to implement the production enviroment.


From those guiding decisions, the rest of the app sort of fell into place. We were more concerned with the functionality of the website than we were with the looks, or replicating the look of the specifications. We decided that function was more important than form, as we could describe the use in the user manual.

\subsection{What can we do on a computer than we can’t do on a printed board?}

\subsection{What is a computerized opponent? What are its objectives?}
A computerized opponent is some logic that can play the game. In the case of connect 4 and toot and otto,
it is essentially trying to minimize a score, and the algorithm considers that to be `winning'.
\url{https://en.wikipedia.org/wiki/Minimax}


\subsection{What design choices exist for the Interface components?}
- Color? Font? Dimensions of Windows? Rescale-ability? Scroll Bars?
We tried to make the website mobile friendly. 
We decided to not use scroll bars at all, because we found it makes the user experience more confusing.
As such, the website supports many different types of dimensions by adjusting the canvas size the game draws.
The colours and fonts were chosen for simplicity
and readability. 

\subsection{What does exception handling mean in a GUI system?}
It means showing a warning to a user in the worst-case scenario. In our design,
we tried to avoid most errors by setting default values to fall back on if something fails. 

\subsection{Do we require a command-line interface for debugging purposes?}

\subsection{What are Model-View-Controller and Model-View-Viewmodel? Are either applicable to your
design?}
Model-View-Controller is a software design pattern, commonly used for user interfaces.
The model contains the data structure, the view takes care of how to display the model,
and the controller handles updates to the model and tells the view when to update.

Model-View-Viewmodel is another software design pattern. The model and view is similar to model-view-controller.
However, the viewmodel meaning the view model is responsible for converting the data objects from the model in such
a way that the objects can be easily presented.

Our application uses the Yew framework, a rust web framework based on Elm, which uses the model
view controller architecture. Our web components have a Model struct, a Message enum, and the
struct implementation has a view() function which returns html elements, while the update()
method determines whether or not the component needs to re-render and how the model should be updated
based on the message passed to it.

\section{Defects}
No known defects.

From a security standpoint, it would be more ideal to serve the site with TLS encryption over HTTPS,
however, we had issues with enabling non-default features in the rocket library.
\url{https://github.com/SergioBenitez/Rocket/issues/975}

We wanted to use rocket\_contrib database pools but that was a later objective and we didn't get to it in the end. As a result, if the database is not running requests to the server API may delay or return partially correct results. For example, if the mongo instance is not running and an user tries to sign in, then it it treated as an invalid password. Attempts were made in a connection\_pool git branch on this project.

The documentation we were following for this feature comes from:
\url{https://api.rocket.rs/v0.4/rocket\_contrib/databases/index.html}
\url{https://rocket.rs/v0.4/guide/state/#databases}

\section{Remaining MEAN stack code}
The only thing left over from the MEAN stack is mongodb. Everything
else has been written in Rust and WebAssembly. In a project with over
5000 lines of code, less than 40 lines are HTML + JavaScript. This is
what is required to load our web assembly.


\section{User Manual}
\markdownInput{../README.md}
\end{document}
